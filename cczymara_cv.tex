\documentclass[11pt, a4paper]{article}
\usepackage{geometry} 
\geometry{a4paper, textwidth=5.5in, textheight=9in, marginparsep=7pt, marginparwidth=1in, right=2.5cm}
\setlength\parindent{0in}

\usepackage{marginnote}

\reversemarginpar
\usepackage{sectsty}

\sectionfont{\mdseries\upshape\Large}
\subsectionfont{\mdseries\scshape\normalsize}
\subsubsectionfont{\mdseries\upshape\large}

\usepackage{titlesec}
\titlespacing{\subsection}{0pt}{\parskip}{-\parskip}

\newcommand{\years}[1]{\marginnote{~~#1}}
\renewcommand*{\raggedrightmarginnote}{}
\setlength{\marginparsep}{7pt}

\renewcommand{\labelitemi}{\textendash} % damit aufzaehlungen nicht mit punkten sondern mit strichen sind

\usepackage{enumitem} % damit kein abstand bei aufzaehlungen
%\setlist[enumerate]{noitemsep, nolistsep} % bei zahlen
\setlist[itemize]{noitemsep,nolistsep} % bei anderen listen

\usepackage[%dvipdfm,
bookmarks, colorlinks, breaklinks,
	pdftitle={Dr. Christian S. Czymara - vita},
	pdfauthor={Dr. Christian S. Czymara},
]{hyperref}
\hypersetup{urlcolor=blue}
\usepackage{fontawesome5}
% \usepackage{academicons} % only works with LuaLatex or XeLatex, which fucks up years
\renewcommand*{\thefootnote}{\color{black}\fnsymbol{footnote}}% fussnote mit sternchen
\title{}
\begin{document}
\author{\huge \textsc{Dr. Christian S. Czymara}}
\date{}
\maketitle
\begin{minipage}{0.5\textwidth}
\href{https://czymara.com}{\faLaptop~czymara.com}\\
\href{mailto:cc@soz.uni-frankfurt.de}{\faIcon[regular]{envelope} cc[at]soz.uni-frankfurt.de}\\
%\href{mailto:c@czymara.com}{{\Letter} c[at]czymara[dot]com}
\href{https://orcid.org/0000-0002-9535-3559}{\faOrcid}
\href{https://scholar.google.de/citations?user=khPqHmgAAAAJ}{\faGoogle}
\href{https://github.com/czymara/}{\faGithub}
%\href{https://methodenderqu-ujz7851.slack.com/team/U010X1DBKMG}{\textcolor[rgb]{0,0,1}{\faSlack}}
\href{https://twitter.com/cczymara}{\faTwitter}
\end{minipage}
\begin{minipage}{0.5\textwidth}
\begin{flushright}
Goethe University Frankfurt\\
Campus Westend -- PEG building\\
office 3.G152
\end{flushright}
\end{minipage}
\section*{\textsc{current position}}
\vskip-20pt{\noindent\rule{\textwidth}{1pt}}
\years{since May 2018}\textit{Postdoctoral researcher and lecturer} at the department for Sociology with Focus on Quantitative Methods for Social Research, Goethe University Frankfurt
\section*{\textsc{research interests}}
\vskip-20pt{\noindent\rule{\textwidth}{1pt}}
inter-group relations; ethnic conflict; public opinion; political communication; computational social science $\|$ quantitative methods for multi-national, longitudinal data \& text corpora
\section*{\textsc{education}}
\vskip-20pt{\noindent\rule{\textwidth}{1pt}}
\noindent
\years{Dec 2018}Dr.~rer.~pol. (\textit{summa cum laude}), Social Sciences, University of Cologne\\[1em]
\years{Aug 2015}M.~Sc., Sociology and Empirical Social Research (minor: Social Policy), University of Cologne\\[1em]
\years{Jun 2012}B.~A., Sociology (minor: Political Science \& Social Psychology), University of Mannheim
%\years{06/2007}Abitur, Friedrich-Schiller-Gymnasium Ludwigsburg
\section*{\textsc{professional experience}}
\vskip-20pt{\noindent\rule{\textwidth}{1pt}}
%\years{since 05/2018}Goethe University Frankfurt\\
%\textbullet~postdoctoral researcher and lecturer at the chair for Sociology with Focus on Quantitative Methods for Social Research\\
\subsection*{prior positions}
\years{Oct 2015 -- Apr 2018}PhD fellow at Cologne Graduate School in Management, Economics and Social Sciences, University of Cologne\\[1em]
\years{Apr 2016 -- Mar 2018}research assistant at Chair for Empirical Social and Economic Research, University of Cologne\\[1em]
\years{Jun 2013 -- Aug 2015}student assistant at Computational Social Science department, GESIS -- Leibniz Institute for the Social Science\\[1em]
\years{May 2010 -- Jul 2012}student assistant at Mannheim Center for European Social Research, project Ethnische und soziale Unterschiede kleinräumlicher Wohnortwahlen (funded by the German Research Foundation)\\[1em]
%\years{08/2007\\to 04/2008}Arbeiter-Samariter-Bund OV Ludwigsburg\\
%\textbullet~community service (Zivildienst)
\subsection*{\textsc{research visits and other jobs}}
\noindent
\years{Aug 2020}Expert Council of German Foundations on Integration and Migration, Annual Report 2021\\[1em]
\years{Feb -- Mar 2018}Radboud University Nijmegen, Department of Sociology\\[1em]
\years{Jun -- Aug 2014}Max Planck Institute for the Study of Societies, Sociology of Markets division\\
\subsection*{selected additional training}
\noindent
\years{2020}GRADE Workshop: Einführung in das maschinelle Lernen\\[1em]
\years{2019}BIGSSS Computational Social Science Summer School on Migration\\[1em]
\years{2017}Oslo Summer School in Comparative Social Science Studies: Collecting and Analyzing Big Data\\
GESIS-Methodenseminar: Text Mining with R
%\years{2016}Zentrum f\"ur Hochschuldidaktik, University of Cologne: tutor training\\
%\years{2013}GESIS Summer School in Survey Methodology: Factorial Survey Designs
\section*{\textsc{publications}}
\vskip-20pt{\noindent\rule{\textwidth}{1pt}}
\subsection*{doctoral thesis}
Czymara, C. S. (2018): Discursive Determinants of Attitudes towards Immigrants: Political Parties and Mass Media as Contextual Sources of Threat Perceptions. Universitäts- und Stadtbibliothek Köln. \href{https://kups.ub.uni-koeln.de/9150}{kups.ub.uni-koeln.de/9150}
\begin{itemize}
	\item with distinction (\textit{summa cum laude})
	\item Deutscher Studienpreis (shortlist), Körber Foundation
\end{itemize}
\hspace{1em}
\subsection*{peer-reviewed journal articles}
\textit{All code for published studies is available at \href{https://osf.io/b3ugm}{osf.io/b3ugm}}\\

\years{9}Czymara, C. S. \& van Klingeren, M. (forthcoming): New perspective? Comparing Frame Occurrence in Online and Traditional News Media Reporting on Europe's ``Migration Crisis''. \textit{Communications: The European Journal of Communication Research}.\\[1em]
\years{8}Czymara, C. S. \& Eisentraut, M. (2020): A threat to the Occident? Comparing human values of Muslim immigrants, Christian and non-religious natives in Western Europe. \textit{Frontiers in Sociology} (5): 1 -- 15. \href{https://doi.org/10.3389/fsoc.2020.538926}{doi.org/10.3389/fsoc.2020.538926}\\[1em]
\years{7}Czymara, C. S., Langenkamp, A. \& Cano, T. (2020): Cause for Concerns: Gender Inequality in Experiencing the COVID-19 Lockdown in Germany. \textit{European Societies}. \href{https://doi.org/10.1080/14616696.2020.1808692}{doi.org/10.1080/14616696.2020.1808692}\\[1em]
\years{6}Czymara, C. S. (2020): Attitudes toward Refugees in Contemporary Europe: A Longitudinal Perspective on Cross-national Differences. \textit{Social Forces}. \href{https://doi.org/10.1093/sf/soaa055}{doi.org/ 10.1093/sf/soaa055}\\[1em]
\years{5}Schmidt-Catran, A. W. \& Czymara, C. S. (2020): ``Did you read about Berlin?'' Terrorist Attacks, Online Media Reporting and Support for Refugees in Germany. \textit{Soziale Welt} 71 (2 -- 3): 305 -- 337. \href{https://doi.org/10.5771/0038-6073-2020-1-2-201}{doi.org/10.5771/0038-6073-2020-1-2-201}\\[1em]
\years{4}Czymara, C. S. (2020): Propagated Preferences? Political Elite Discourses and Europeans’ Openness toward Muslim Immigrants. \textit{International Migration Review} 54 (4): 1212 -- 1237. \href{https://doi.org/10.1177/0197918319890270}{doi.org/10.1177/0197918319890270}\\[1em]
\years{3}Czymara, C. S. \& Dochow, S. (2018): Mass Media and Concerns about Immigration in Germany in the 21st Century: Individual-Level Evidence over 15 Years. \textit{European Sociological Review} 34 (4): 381 -- 401. \href{https://doi.org/10.1093/esr/jcy019}{doi.org/10.1093/esr/jcy019}\\[1em]
\years{2}Czymara, C. S. \& Schmidt-Catran, A. W. (2017): Refugees Unwelcome? Changes in the Public Acceptance of Immigrants and Refugees in Germany in the Course of Europe's ``Immigration Crisis''. \textit{European Sociological Review} 33 (6): 735 -- 751. \href{https://doi.org/10.1093/esr/jcx071}{doi.org/10.1093/esr/jcx071}
\begin{itemize}
	\item Best Paper Award, Bamberg Graduate School of Social Sciences
	\item Early Career Award (honorable mention), European Survey Research Association
	\item among ESR's \href{https://academic.oup.com/esr/pages/Top_Cited_Papers}{``top cited papers''}
\end{itemize}
\hspace{1em}

\years{1}Czymara, C. S. \& Schmidt-Catran, A. W. (2016): Wer ist in Deutschland willkommen? Eine Vignettenanalyse zur Akzeptanz von Einwanderern (Who is welcome in Germany? A Vignette Study on the Acceptance of Immigrants). \textit{K\"olner Zeitschrift f\"ur Soziologie und Sozialpsychologie} 68 (2): 193 -- 227. \href{https://doi.org/10.1007/s11577-016-0361-x}{doi.org/10.1007/s11577-016-0361-x}
\begin{itemize}
	\item Preis der Fritz Thyssen Stiftung für sozialwissenschaftliche Aufsätze
\end{itemize}
\hspace{1em}

\subsection*{research notes \& working papers}
\years{4}Langenkamp, A., Cano, T. \& Czymara, C. S. (2020): My home is my castle? The role of living arrangements on experiencing the COVID-19 pandemic in Germany. \textit{SocArXiv}. \href{https://osf.io/preprints/socarxiv/6c42q/}{osf.io/preprints/socarxiv/6c42q}\\[1em]
\years{3}Breznau, N. et al. (2019): The Crowdsourced Replication Initiative: Investigating Immigration and Social Policy Preferences using Meta-Science. Executive Report. \textit{SocArXiv}. \href{https://osf.io/preprints/socarxiv/6j9qb}{osf.io/preprints/socarxiv/6j9qb}\\[1em]
\years{2}Czymara, C. S. \& Schmidt-Catran, A. W. (2018): Konfundierungen in Vignettenanalysen mit einzelnen d-effizienten Vignettenstichproben (Confounding in Vignette Studies with Single D-Efficient Vignette Samples). \textit{K\"olner Zeitschrift f\"ur Soziologie und Sozialpsychologie} 70 (1): 93 -- 103. \href{https://doi.org/10.1007/s11577-018-0516-z}{doi.org/10.1007/s11577-018-0516-z}\\[1em]
\years{1}Czymara, C. S. (2014): How do Economic Wealth and the Relative Group-Size of Immigrants Affect Natives' Attitudes toward Immigration? \textit{Unpublished manuscript}
\begin{itemize}
	\item Janet A. Harkness Student Paper Award, World Association for Public Opinion Research and American Association for Public Opinion Research
\end{itemize}
\section*{\textsc{selected presentations}}
\vskip-20pt{\noindent\rule{\textwidth}{1pt}}
\subsection*{conferences}
\noindent
\years{2019}Academy of Sociology, Constance, DE\\
European Consortium for Political Research, Wroclaw, PL\\
European Sociological Association, Manchester, UK\\
European Survey Research Association, Zagreb, HR\\
European Social Survey, Mannheim, DE\\
International Sociological Association RC28 Spring Meeting, Frankfurt, DE\\
Etmaal: Communication Science and Artificial Intelligence, Nijmegen, NL\\
\years{2018}GESIS-IEDI Tagung zu Migration und interethnischen Beziehungen, Cologne, DE\\
World Association for Public Opinion Research, Marrakesh, MA\\
Ruppin International Conference on Immigration and Social Integration, Ruppin Academic Center, IL\\
\years{2017}%ISS Forum, Cologne, Germany\\
Bamberg Graduate School of Social Sciences, Bamberg, DE\\
European Survey Research Association, Lisbon, PT\\
%Interdisciplinary Workshop on Opinion Dynamics and Collective Decision, Bremen, Germany (poster pres. w/ S. Dochow)\\
%\years{2016}ISS Forum, Cologne, Germany\\
%\years{2014}World Association for Public Opinion Research, Nice, France
\subsection*{invited talks}
\noindent
\years{2019}%Deutscher Studienpreis, Körber Stiftung, Berlin, DE\\
Forum Methodenzentrum: Data Science (poster pres.), Goethe University Frankfurt, DE\\
\years{2018}Global Refugee and Migration in the Twenty-first Century, Georgetown University, Washington, D.C., US\\
%InFER Colloquium, Goethe University Frankfurt, Germany\\
Seminar Sociology, University of Tilburg, NL\\
Research Seminar, Radboud University Nijmegen, NL
%\years{2017}SOCLIFE Research Seminar, University of Cologne, Germany
\section*{\textsc{awards \& grants}}
\vskip-20pt{\noindent\rule{\textwidth}{1pt}}
\subsection*{awards}
shortlist of German thesis award (\href{https://www.koerber-stiftung.de/fileadmin/user_upload/koerber-stiftung/redaktion/deutscher-studienpreis/pdf/2019/2019_Nominierte_SozialWissenschaften.pdf}{Deutscher Studienpreis}), Körber Foundation\\
best paper award, Bamberg Graduate School of Social Sciences\\
honorable mention for \href{https://www.europeansurveyresearch.org/awards/prize}{early career award}, European Survey Research Association\\
best paper award (\href{https://www.fritz-thyssen-stiftung.de/cms/wp-content/uploads/2018/06/Jahresbericht_2017_interaktiv.pdf}{Preis für sozialwissenschaftliche Aufsätze}), Fritz Thyssen Foundation\\
\href{https://wapor.org/events/annual-conference/awards-funds/janet-a-harkness-student-paper-award/}{Janet A. Harkness student paper award}, World Association for Public Opinion Research and American Association for Public Opinion Research\\
\subsection*{grants}
travel grant ESRA meeting (Zagreb, HR), German Academic Exchange Service\\
participation BIGSSS Summer School (Sardinia, IT), Volkswagen Foundation\\
three-year Ph.~D. fellowship, Cologne Graduate School in Management, Economics and Social Sciences
\section*{\textsc{teaching}}
\vskip-20pt{\noindent\rule{\textwidth}{1pt}}
\noindent
\subsection*{Goethe University Frankfurt}
\years{SS 2020}Längsschnittdatenanalyse und Kausalität (digital \textsc{Covid}-19 edition): \href{https://czymara.com/wp-content/uploads/2020/08/Profillinie-Digitales_Forschungspraktikum_I_und_II__Langsschnittdatenanalyse_und_Kausalitat.pdf}{1.4/6}\footnote{1: very good}\\
\years{WS 2019/20}Vergleichende Sozialforschung mit Mehrebenenmodellen (4 hours/week): \href{https://czymara.com/wp-content/uploads/2020/07/Profillinie-Vergleichende_Sozialforschung_mit_Mehrebenenmodellen.pdf}{1.5/6}\textsuperscript{*}\\
\years{SS 2019}Analyzing longitudinal data and the issue of causality (4 h): \href{https://czymara.com/wp-content/uploads/2020/07/Profillinie-Research_Training_Part_I___Analyzing_Longitudinal_Data.pdf}{1.4/6}\textsuperscript{*}\\
\years{WS 2018/19}Quantitative comparative social research with multi-level modeling (4 h): \href{https://czymara.com/wp-content/uploads/2020/07/WS18_19-Profillinie-Quantitative_comparative_social_research_with_multi-level__modeling.pdf}{1.6/6}\textsuperscript{*}\\
\years{SS 2018}An applied introduction into quantitative comparative social research (block): 1.7/5\textsuperscript{*}\\
\subsection*{University of Cologne}
\years{WS 2017/18}Analysis of cross-sectional data (as tutor): \href{https://czymara.com/wp-content/uploads/2018/03/eval-analysis_of_crosssectional_data-cc_ws1718.pdf}{1.4/5}\textsuperscript{*}\\
\years{SS 2017}Analysis of longitudinal data (as tutor): no evaluation\\
\years{WS 2016/17}Analysis of cross-sectional data (as tutor): \href{https://czymara.com/wp-content/uploads/2017/02/eval-analysis_of_crosssectional_data-cc_ws1617.pdf}{1.6/5}\textsuperscript{*}\\
\years{SS 2016}Analysis of longitudinal data (as tutor): \href{https://czymara.com/wp-content/uploads/2016/09/eval-analysis_of_longitudinal_data-cc_ss16.pdf}{2.0/5}\textsuperscript{*}
\section*{\textsc{professional service}}
\vskip-20pt{\noindent\rule{\textwidth}{1pt}}
\noindent
\subsection*{reviewing}
journals: $>$35 reviews for European Sociological Review, Social Forces, Social Psychology Quarterly, Social Science Research, Acta Sociologica, Journal of Ethnic and Migration Studies, Ethnic and Racial Studies, International Journal of Comparative Sociology, European Societies, Journal of European Social Policy, European Union Politics, Political Studies, Kölner Zeitschrift für Soziologie und Sozialpsychologie, Jahrbücher für Nationalökonomie und Statistik; also see \href{https://publons.com/researcher/3006411/christian-s-czymara/}{publons.com/researcher/3006411}\\

grant proposals: VolkswagenStiftung; National Science Center Poland; PSC-City University of New York Research Award Program\\

conferences: ISA RC28 Spring Meeting 2019; IAB-ECSR Refugee Conference 2020\\

conference discussant: How should we analyze country differences and what have we learned about them? Analytical strategies and explanations based on international comparative surveys, University of Cologne 2017\\
\subsection*{current memberships}
International Sociological Association; European Sociological Association; European Survey Research Association; ISA Research Committee 28 on Social Stratification and Mobility; ECPR Standing Groups: Political Methodology, Political Communication, Migration and Ethnicity; Akademie für Soziologie\\
\subsection*{other}
Mittelbauvertreter (representative) of Institute of Sociology at Goethe University Frankfurt (WS 18/19 \& SS 19)
%\section*{\textsc{skills}}
%\noindent
%\years{software \&\\programming}R, Stata, \LaTeX~/ \textsc{Bib}\TeX, Libre- \& MS-Office -- advanced\\
%HTML / CSS / PHP, Adobe Photoshop / GIMP -- good\\
%Python, QGIS, MAXQDA -- basics\\[1ex]
%\years{\normalsize{languages}}English -- fluent\\
%Latinum\\
%German -- mother tongue\\[2ex]
%\years{web presence}\href{https://czymara.com/}{my homepage}\\
%\href{https://osf.io/b3ugm/}{Open Science Framework}\\
%\href{https://scholar.google.de/citations?user=khPqHmgAAAAJ}{Google Scholar}\\
%\href{https://www.researchgate.net/profile/Christian_Czymara}{ResearchGate}\\
%\href{https://publons.com/researcher/3006411/christian-s-czymara/}{Publons}%\\[1cm]
%\years{\normalsize{Born}}March 1988 -- Ludwigsburg, Germany
\vfill{}
%\hrulefill
last updated: \today
\end{document}